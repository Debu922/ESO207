\documentclass[10pt, a4paper]{article}

% Formatting

\usepackage[utf8]{inputenc}
\usepackage[a4paper, top=1cm, left = 2cm, right = 2cm, bottom = 2cm]{geometry}
\usepackage[titletoc, title]{appendix}

\usepackage{amsmath, amsfonts, amssymb, mathtools}
\usepackage{graphicx, float}
\usepackage{algorithm}
% \usepackage{arevmath}     % For math symbols
\usepackage[noend]{algpseudocode}

\newcommand{\get}{\leftarrow}
\newcommand{\BigO}{\mathcal{O}}
% \newcommand{\Return}{\textbf{Return}}
\usepackage{listings}
\usepackage{xcolor}
%New colors defined below
\definecolor{codegreen}{rgb}{0,0.6,0}
\definecolor{codegray}{rgb}{0.5,0.5,0.5}
\definecolor{codepurple}{rgb}{0.58,0,0.82}
\definecolor{backcolour}{rgb}{0.95,0.95,0.92}

%Code listing style named "mystyle"
\lstdefinestyle{mystyle}{
  backgroundcolor=\color{backcolour},   commentstyle=\color{codegreen},
%   keywordstyle=\color{magenta},
  numberstyle=\tiny\color{codegray},
%   stringstyle=\color{codepurple},
  basicstyle=\ttfamily\footnotesize,
  breakatwhitespace=false,         
  breaklines=true,                 
  captionpos=b,                    
  keepspaces=true,                 
  numbers=left,                    
  numbersep=5pt,                  
  showspaces=false,                
  showstringspaces=false,
  showtabs=false,                  
  tabsize=2
}

\lstset{style=mystyle}
\title{ESO207: Theoretical Assignment 1 - Part 1}
\author{Debaditya Bhattacharya}
\date{\today}

\begin{document}
\maketitle

\noindent\begin{minipage}{2in}
    \textbf{Name: }Debaditya Bhattacharya \\
    \textbf{Email id: }debbh@iitk.ac.in  
\end{minipage}
\hfill
\noindent\begin{minipage}{1.9in}
    \textbf{Roll No: }190254\\
    \textbf{Hackerrank Id: }debbh922
\end{minipage}

\noindent\rule{\textwidth}{1px}
\parskip 0.3cm

% ----------- Problem 2 -----------
\noindent\large{\textbf{Problem 2}}
\vspace{5pt}

\normalsize

\noindent\textbf{Description:}

Let $r = k\mod(q-p)$ (To rotate properly when $k>q-p$). If $r=0$ no modification needed, return $head$. If $r\neq 0$ Iterate through the list to get $a_p$, $a_q$ and $a_{q-r}$ $(T =\BigO(1))$. We will now work with the sub-list $a_p$ \dots $a_q$ Connect $a_p$ and $a_q$ to form circular loop. $a_{q-r+1}$ takes first position in the sub array. $a_{q-r}$ takes last position. Adjust all linking paths appropriately.

\vspace{5pt}

\noindent\textbf{Algorithm:}
\begin{lstlisting}[language=C++,caption=Rotate sublist]
    function rotate_sublist(head, p, q, k)
    {
        r <- k%(p-q);

        if (r == 0);                    \\ No rotation needed
            return head;

        ptr <- head;
        for(i = 1; i <= q; i++)         \\ O(log(q)) time to fetch required pointers
        {  
            if (p == i)
                ptr_p <- ptr;           \\Pointer to p element
            if (q == i)
                ptr_q <- ptr;           \\Pointer to q element
            if (q-r == i)
                ptr_q-r <- ptr;         \\Pointer to q-r element
            if (i != q)
                ptr <- ptr.right;
        }
        if(ptr_p.left == NULL)          \\p was head
            ptr_p-1 <- NULL;
        else
            ptr_p-1 <- ptr_p.left;
    
        
        if(ptr.q.right == NULL)         \\q was tail
            ptr_q+1 <- NULL;
        else
            ptr_q+1 <- ptr_q.right;

        ptr_p.left <- ptr_q;            \\Update pointers
        ptr_q.right <- ptr_p;
        ptr_q-r+1.left <- ptr_p-1;
        ptr_q-r.right <- ptr_q+1;
        if(ptr_p-1 != NULL)
            ptr_p-1.right <- ptr_q-r+1;
        if(ptr_q+1 != NULL)
            ptr_q+1.left <- ptr_q-r;
        
        return head;
    }
\end{lstlisting}

\noindent\rule{\textwidth}{1px}

% ---------------------------------
\newpage
% ----------- Problem 3 -----------
\noindent\large{\textbf{Problem 3}}
\vspace{10pt}

\normalsize

\noindent\textbf{Description:}

In merged and sorted array $C$, the elements $C[i]$ such that $i<n$ will contain elements from $A$ or $B$. If $C[i]$ contains the first $r$ elements of $A$ then it must contain the first $n-1-r$ elements of B. To find $r$ we impose the condition $(1)$ $A[r]>B[n-1-r]$ and condition $(2)$ $A[r-1] < B[n-1-r+1]$. If (1) is false ignore the left side of L and divide. If $(1)$ is true, and $(2)$ is false ignore the right hand side. If Both $(1)$ and $(2)$ is median is $\frac{(max([A[r-1], B[n-1-r]]) + min([A[r], B[n-r]]))}{2}$. 

\begin{center}
    \begin{tabular}{ |c||c|c|c|c|c| }
     \hline
     index & 0 & 1 & 2 & 3 & 4\\
     \hline
     A& 12* & 17* & 23* & 34* & 65\\
     \hline
     B& 40* & 53 & 59 & 61 & 66\\
     \hline
     \hline
     A& 12* & 17* & 23* & 34* & 65\\
     \hline
     B'& 66 & 61 & 59 & 53 & 40*\\
     \hline
     index & 4 & 3 & 2 & 1 & 0\\
     \hline
    \end{tabular}
\end{center}

To understand better consider the example above, where we have flipped $B$ to $B'$ for easier understanding. We need to find compare $23$ and $59$. $23 < 59$ therefore we ignore the left side of $23$. We then move to the next middle element $34$. As $34<53$ we ignore to left of $34$. Once at $65$, (1) and (2) are met. therefore median is $\frac{max(34, 40) + min(53,65)}{2}=(40+53)/2 = 46.5$.

\vspace{10px}
\normalsize
\noindent\textbf{Algorithm:}
\begin{lstlisting}[language=C++,caption=Find median]
    function find_median(A, B, n){
        if (A[0] > B[n-1])                              \\Trivial case
            return (B[n-1] + A[0])/2.0;
        if (A[n-1] < B[0])                              \\Trivial case
            return (A[n-1] + B[0])/2.0;
        L <- 0;                                         \\Boundary control
        R <- n-1;
        idx = -1;
        while (idx == 0){
            m = (L+R)/2;                                \\Middle element
            if (A[m] > B[n-1-m]){                       \\check Condition (1)
                if (A[m-1] < B[n-1-m+1]){               \\check Condition (2)
                    idx = m;                            \\Both Condition (1) and (2) true
                    break;
                }
                R = m-1;                                \\Condition (1) but not (2)
            }
            else{
                L = m+1;                                \\Neither (1) nor (2)
            }
        }
        median <- (max([A[idx-1], B[n-1-idx]]) + min([A[idx], B[n-1-idx+1]))/2;
        return median;
    }
\end{lstlisting}

\vspace{10px}
\noindent\textbf{Proof of correctness:}

Trivial cases are when first $n-1$ elements of the merged array are either completely $A$ or $B$. Mode is then just the (last element of first array + first element of last array)/2.

Assertion: P(i): There is a turning point(k such that condition (1) $A[k] > B[n-1-k]$ and condition (2) $A[k-1] < B[n-k]$) for $ L \leq k \leq R$ hold. (Interpret as as point k such that in $A$, for all points before k, is before the median, and for $B$ all points before $n-1-k$ are is before the median.)

Base Case: $L = 0,R=n-1$. As $A[0] \ngtr B[n-1]$ and $A[n-1] \nless B[0]$, we will exist $k$ such that $P(1)$ is true, as we have excluded the trivial cases.

Induction case: Let $P(i)$ be true. Then consider the case $P(i+1)$, take $m = (L+R)/2$. If (1) does not hold for $A[m]$ then (1) does not hold for all $k<m$ because the array A is increasing. Hence we can set $L = m+1$. If condition (1) holds for m, and condition (2) does not hold, then we can say that (2) does not hold for all $k>m$ because the array A is increasing. We can then set $R = m-1$. If both (1) and (2) hold. We have found the turning point k.

Median: At the turning point the $C[n-1], C[n]$ can be found as we know that $C[n-2], C[n-1], C[n], C[n+1]$ terms are $A[k-1], B[n-1-k], A[k], B[n-k]$ in unsorted order. We can then find the median of this 4 element array in O(1) time. this is the median of the larger array.

\noindent\rule{\textwidth}{1px}
% ----------- Problem 4 -----------
\noindent\large{\textbf{Problem 4}}
\vspace{5pt}
\normalsize

\vspace{10px}
\noindent\textbf{Description:}

\noindent First we reduce create an auxiliary array $X$ using the relation $X[i] = \frac{a^2-cA[i]}{b}$. We note that as $A[i]$ was sorted in increasing order, $X[i]$ will be in decreasing. We then start comparing the elements of $X[i]$ and $A[j]$(from the back), increasing $i$if $X[i]$ is smaller than $A[j]$ and decreasing j if $A[j]$ is smaller. In this way we traverse and compare the all the elements in $T=\BigO(2n)$, in the worst case. If $X[i] = A[j]$, return j, else at the end return -1.

\vspace{10px}
\noindent\textbf{Algorithm:}
\begin{lstlisting}[language=c++,caption=checkXY]
    function checkXY(A, n, a, b, c){
        X <- empty array of floats array of size A.size;

        for (i = 0; i<n; i++){                      \\T(n) = O(n)
            X[i] = (a*a - c*A[i])/b;                \\Reduce to comparable form.
        }
        
        i <- 0;
        j <- n-1;
        while ((i<n) && (j>=0)){                    
            if(X[i]==A[j]){
                if (!i = n-1-j){                    \\Since distinct x,y
                    return 1;                       \\Found element 
                }     
            }
            if(X[i]<A[j]){
                j--;
            } else {
                i++;
            }
        }

        return -1                                   \\Not found, does not exist.  
    }
\end{lstlisting}

\vspace{10px}
\noindent\textbf{Proof of correctness:}

For the relation $a^2 = bx +cy$ to hold for distinct $x,y\in A$. It is sufficient to show that if $y\in A$ then $x$ must be of the form $x = \frac{a^2-cy}{b}$.

We may find possible values of $X[i] = \frac{a^2-cA[i]}{b}$ through an iterative manner in $T = \BigO(n)$. As $A$ is sorted in increasing order (Assume without loss of generality), $X$ will be sorted in decreasing order. We proceed to find comparing $X$ and $A$ to find the any common elements.

As the arrays are sorted in opposite order, we start comparing the values of $X$ from the start and $A$ from the end. If $X[i]<A[j]$, we shift then shift our focus to $A[j-1]$, and vice versa. If $X[i]=A[j]$ and $ i\neq n-1-j$ (Since x, y need to be distinct). At each iteration, we will shift one of $i$ or $j$. Hence we will visit each element at most once. The worst case time taken will be $T = \BigO(2n) = \BigO(n)$.

Hence the entire algorithm wil run in $T = \BigO(n)$, and will return 1 if such $x,y \in A$ and will return 0 if it is not possible.

\noindent\rule{\textwidth}{1px}
\newpage

% ----------- Problem 1 -----------
\noindent\large{\textbf{Problem 1}}
\vspace{5pt}
\normalsize

\noindent\textbf{Description:}

We assume that we have $n$ computers at our disposal. Computers are labeled $0,1,2\dots n$. Computer $i$ compares $A[i-1],A[i],A[i+1]$. If A is a local maximum return i, else do not return anything. The driver CPU will now receive 2 values of $i$ from two computers. These are the required data points. This can be done in $T=\BigO(1)=\BigO(\log n)$

\noindent\textbf{Algorithm:}
\begin{lstlisting}[language=C++,caption=Find two maximas.]
    function check_maxima(A,n){
        i <- SELF_LABEL                  \\Predefined macro available to the computer.
        if (i == 0){
            if (A[i]>A[i+1])
                return i;               \\First element is maxima, return 0
            else
                return;
        }
        if (i == n-1){
            if (A[i]>A[i-1])            \\Last element is maxima, return n-1
                return i;
            else
                return;
        }
        if (A[i]>A[i-1] && A[i]>A[i+1]) \\Element is maximum, return i
            return i;
        else
            return;
    }
\end{lstlisting}

\noindent\rule{\textwidth}{1px}

\end{document}
